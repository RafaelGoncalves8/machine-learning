\documentclass[a4paper, 12pt]{article}

\usepackage[portuges]{babel}
\usepackage[utf8]{inputenc}
\usepackage[margin=1.2in]{geometry}
\usepackage{amsmath}
\usepackage{graphicx}
\usepackage{datetime}
\usepackage{enumerate}
\renewcommand{\baselinestretch}{1.5}

\emergencystretch 1em%

\begin{document}

\setlength{\parindent}{0pt}

\section*{Parte I - Atividades teóricas}

\subsection*{Exercício 1}

\begin{enumerate}[a)]
\item
$P(A^C) = 1 - P(A) = 1 - \frac{1}{3} = \frac{2}{3}$

\item
$P(A^C \cup B) = P(A^C) + P(B) - P(A^CB) = P(A^C) + P(B) - (P(B) - P(AB)) = P(A^C) + P(AB) = \frac{2}{3} + \frac{1}{6} = \frac{5}{6}$

\item
$P(A \cup B^C) = P(A) + P(B^C) - P(AB^C) = P(B^C) + P(AB) = (1 - P(B)) + P(AB) = (1 - \frac{1}{4}) + \frac{1}{6} = \frac{11}{12}$

\item
    $P(AB^C) = P(A) + P(B^C) - P(A \cup B^C) = \frac{1}{3} + \frac{3}{4} - \frac{11}{12} = \frac{2}{12} = \frac{1}{6}$

\item
    $P(A^C \cup B^C) = 1 - P(AB) = 1 - \frac{1}{6} = \frac{5}{6}$

\end{enumerate}

\subsection*{Exercício 2}

\begin{enumerate}[a)]
\item
$$
    F_X(x) = P(X \leq x) = \int\limits_{-\infty}^x f_X(\xi) d\xi = \int\limits_{-\infty}^x \frac{1}{2} d\xi = \left[ \frac{1}{2}\xi \right]^x_0 = \frac{1}{2} x, \forall X \in [0, 2]
$$

\item
$E\{X\} = \int\limits_{-\infty}^{\infty} xf_X(x)dx = \int\limits_0^2 \frac{1}{2}x dx = \left[\frac{x^2}{4}\right]^2_0 = 1$

$E\{X^2\} = \int\limits_{-\infty}^{\infty} x^2f_X(x)dx = \int\limits_0^2 \frac{1}{2}x^2 dx = \left[\frac{x^3}{6}\right]^2_0 = \frac{4}{3}$

$E\{X^3\} = \int\limits_{-\infty}^{\infty} x^3f_X(x)dx = \int\limits_0^2 \frac{1}{2}x^3 dx = \left[\frac{x^4}{8}\right]^2_0 = 2$
\end{enumerate}

\subsection*{Exercício 3}

\begin{enumerate}[a)]
\item
$X_2$, pois quanto mais próximo da distribuição uniforme, mais difícil é acertar o resultado de um evento aleatório "chutando um valor", ou seja, a variável aleatória $X_2$ carrega mais informação que a variável aleatória $X_1$, pois neste último eu poderia supor que o resultado será sempre 3 com uma taxa de acertos de 40\% contra uma taxa de acertos de 25\% para qualquer valor estimado para $X_2$ se não tivermos nenhuma informação \em a priori\em .


\item
$H(X_1) = - \sum\limits_x p(X_1)log_2[p(X_1)] = -[0,1(-3,32) + 0,2(-2,32) + 0,3(-1,74) + 0,4(-1,32)] = 1,85$

$H(X_2) = - \sum\limits_x p(X_2)log_2[p(X_2)] = - [0,25 (-2) + 0,25 (-2) + 0,25 (-2) + 0,25(-2)] = 2$


\item
$D(P_1 || P_2) = \sum\limits_x p(X_1)log_2\left[\frac{p(X_1)}{p(X_2)}\right] = 0,1(-1,32) + 0,2(-0,32) + 0,3(-0,26) + 0,4(-0,68) = -0,54$

$D(P_2 || P_1) = \sum\limits_x p(X_2)log_2\left[\frac{p(X_2)}{p(X_1)}\right] = 0,25(1.32 +  0.32 - 0.26 -0.68) = 0,18$

\end{enumerate}

\subsection*{Exercício 4}

\begin{enumerate}[a)]
\item
$\mu_{ML} = arg max_\mu p(x|\mu) = arg max_\mu log[p(x|\mu)] = arg max_\mu \frac{p(x\mu)}{p(\mu)} = x$

\item
$\mu_{ML} = arg max_\mu p(\mathbf{x}|\mu) = arg ax_\mu log[p(\mathbf{x}|\mu)] = arg max_\mu \sum\limits_{k=1}^{N} log[p(x_k|\mu)] = arg max_\mu \sum \limits_{k=1}^{N} log\left [ \frac{p(x_k\mu)}{p(\mu)}\right ] $

\item

    $\mu_{ML} = \frac{1}{N} \sum \limits_{k=1}^{N} x_k$

\end{enumerate}

\newpage

\section*{Parte II - Atividade computacional}

\subsection*{a) Solução fechada com MMQ}

$
\mathbf{X} =\begin{bmatrix}
    \mathbf{x_1} \\
    \mathbf{x_2} \\
    \vdots \\
    \mathbf{x_N}
\end{bmatrix}
$
\hspace{2em}
$
\boldsymbol{\Phi} = \boldsymbol{\Phi}(\mathbf{X}) = \begin{bmatrix}
    1 & \mathbf{x_1} \\
    1 & \mathbf{x_2} \\
    &\vdots \\
    1 & \mathbf{x_N}
\end{bmatrix}
$
\hspace{2em}
$
\mathbf{y} =\begin{bmatrix}
    y_1 \\
    y_2 \\
    \vdots \\
    y_N
\end{bmatrix}
$

\vspace{1em}

Modelo:
\begin{equation}
    \hat{y} = \hat{y}(\mathbf{X}) = \boldsymbol{\Phi}^T\mathbf{w}
\end{equation}

Solução ótima com o método de mínimos múltiplos quadrados (MMQ):
\begin{equation}
    \mathbf{w} = (\boldsymbol{\Phi}^T\boldsymbol{\Phi})^{-1}\boldsymbol{\Phi}^T\mathbf{y}
\end{equation}

Raiz quadrada do erro quadrático médio (RMSE) para o modelo treinado com o conjunto de treino:

$ RMSE _{train} = 15.3702$
\hspace{2em}
$ RMSE _{test} = 14.2495$

Resultado no conjunto de test:

\begin{figure}[h!]
    \centering
    \includegraphics[width=10cm]{images/raw.png}
    \caption{Saída prevista pelo modelo (azul) em comparação com valores reais dos últimos 5 anos (60 meses).}
\end{figure}

\subsection*{b) Seleção de atributos usando wrapper (backward elimination), validação cruzada (k-fold) e regularização L2}

\begin{figure}[h!]
    \centering
    \includegraphics[width=10cm]{images/backward.png}
    \caption{Menor RMSE para cada passo do wrapper (atributo retirado).}
\end{figure}

Modelo escolhido pelo wrapper utilizando k-fold com 5 pastas:

Número de atributos removidos: $10 \quad\quad \lambda: 70 \quad\quad \text{RMSE: } 14.1138$

Atributos escolhidos: 0 (bias) , 1, 2, 3, 4, 5, 6, 9, 11, 18, 20

\begin{figure}[h!]
    \centering
    \includegraphics[width=10cm]{images/wrapper.png}
    \caption{Saída prevista pelo modelo (azul) em comparação com valores reais dos últimos 5 anos (60 meses) com seleção de atributos usando wrapper.}
\end{figure}


min RMSE = 15.4457249245 | lambda = 70

min RMSE = 15.4368725061 | lambda = 70

min RMSE = 15.4318450853 | lambda = 70

min RMSE = 15.4287714380 | lambda = 70

min RMSE = 15.4261395721 | lambda = 70

min RMSE = 15.4243489321 | lambda = 70

min RMSE = 15.4231081954 | lambda = 70

min RMSE = 15.4214290196 | lambda = 70

min RMSE = 15.4202697965 | lambda = 70

min RMSE = 15.4200527097 | lambda = 70

min RMSE = 15.4212809339 | lambda = 70

min RMSE = 15.4263120869 | lambda = 70

min RMSE = 15.4390084229 | lambda = 60

min RMSE = 15.4890363689 | lambda = 60

min RMSE = 15.5561039311 | lambda = 60

min RMSE = 15.7349123813 | lambda = 60

min RMSE = 16.0097566606 | lambda = 70

min RMSE = 16.0799828192 | lambda = 50

min RMSE = 16.8712573347 | lambda = 60

\begin{table}[]
\begin{tabular}{lll}
N atributos retirados & min RMSE & lambda \\
0                     &          & 70     \\
1                     &          & 70     \\
2                     &          & 70     \\
3                     &          & 70     \\
4                     &          & 70     \\
5                     &          & 70     \\
6                     &          & 70     \\
7                     &          & 70     \\
8                     &          & 70     \\
9                     &          & 70     \\
10                    &          & 70     \\
11                    &          & 70     \\
12                    &          & 70     \\
13                    &          & 60     \\
14                    &          & 60     \\
15                    &          & 60     \\
16                    &          & 60     \\
17                    &          & 70     \\
18                    &          & 50     \\
19                    &          & 60    
\end{tabular}
\end{table}


\subsection*{c) Seleção de atributos usando filtro (correlação de Pearson)}

\begin{figure}[h!]
    \centering
    \includegraphics[width=8cm]{images/corr.png}
    \caption{Matriz de correlação, onde uma cor mais avermelhada significa uma correlação maior (dados foram alinhados na matriz de forma que a primeira linha e coluna correspondem ao rótulo e as próximas linhas e colunas correspondem aos atributos - mês anterior até 20 meses atrás).}
\end{figure}

Atributos selecionados: 0 (bias) , 1, 2, 3, 4, 5, 6, 7, 8, 9, 10

$\lambda: 30 \quad\quad RMSE: 13.8698$

\begin{figure}[h!]
    \centering
  \includegraphics[width=10cm]{images/filter.png}
    \caption{Saída prevista pelo modelo (azul) em comparação com valores reais dos últimos 5 anos (60 meses) com seleção de atributos usando filtro.}
\end{figure}

\begin{figure}[h!]
    \centering
  \includegraphics[width=10cm]{images/comparison.png}
    \caption{Saída prevista por cada modelo: azul - modelo inicial com 20 atributos, RMSE = 14.2495; amarelo - modelo criado usando wrapper, RMSE = 14.1138; verde - modelo criado usando filtro, RMSE = 13.8698;  em comparação com valores reais (vermelho) dos últimos 5 anos (60 meses).}
\end{figure}

\end{document}
